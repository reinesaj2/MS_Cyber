\documentclass[12pt]{article}
\usepackage{geometry}
\geometry{margin=1in}
%\usepackage{amsmath amssymb}
\usepackage{graphicx}
\usepackage{hyperref}
\usepackage{fancyhdr}
\usepackage{enumitem}
\usepackage{setspace}
\usepackage{hyperref}
\usepackage{xcolor}

\title{Ethical Analysis of Facebooks Data Practices}
\author{Abraham J. Reines}
\date{\today}

\begin{document}

\maketitle

\begin{abstract}
This paper analyzes ethical implications of Facebooks data practices during 2016 presidential election campaign. Using five leading theories of computer ethics we examine contradiction between Facebooks mission as a social platform and its revenue-generating practices.
\end{abstract}

\section{Introduction}
2016 presidential election highlighted significant issues with Facebooks data practices. Cambridge Analyticas involvement in data mining and targeted advertising campaigns raised concerns about user privacy. Mark Zuckerbergs testimony before United States Congress further revealed tension between Facebooks public commitments and its operationsf. This paper explores ethical implications of these practices focusing on contradiction between Facebooks mission and its profit motives.

\section{Fundamental Issues with Social Media Platforms}
Social media platforms including Facebook often lack a concise organisational mission statement which aligns with ethical principles. Facebooks dual role as social platform and profit generator creates a conflict between user privacy and revenue generation. pursuit of profit can lead to practices which compromisex user data raising ethical concerns about user trust and corporate responsibility.

\section{Ethical Analysis Using Five Theories}
\subsection{Consequentialism}
Consequentialism evaluates actions based on their outcomes. Facebooks data practices reveal consequences for user privacy. Exploitation of user data for targeted advertising contributed to misinformation and/or public distrust in social media platforms. Overall societal impact includes increased polarization and erosion of democratic processes.

\subsection{Virtue and Moral Duties}
Virtue ethics focuses on character and moral duties of individuals and organizations. Facebooks data practices which prioritize profit over ethical behavior reflect a lack of virtuous conduct. Ethical virtues liek honesty integrity and responsibility are compromised by user data bieing exploited for financial gain. Facebook has a moral duty to protect user privacy and act in best interest of its users being a social media powerhouse with extreme influence over societal normalization.

\subsection{Conflict Perspective}
Conflict perspective examines power dynamics and inherent conflicts within social structures. Facebooks data practices highlight power imbalance between corporation and its users. Users have limited control over their data while Facebook wields significant influence over information dissemination and user behavior. This power dynamic fosters conflicts of interest where corporate profits take precedence over user rights and ethical considerations.

\subsection{Social Contract Theory}
Social contract theory posits which individuals and organizations enter into agreements for mutual benefits. Facebooks relationship with its users has an implied contract where users provide data in exchange for access to platform.  Facebooks exploitation of this data breaches social contract by undermining user trust and ultimately violating privacy. Ethical practices require honoring terms of this implicit agreement.  It is ethical to honor an agreement because:  it upholds the mutual terms of cooperation essential for maintaining social order.  This contradicts the overall goal of a massive billion dollar compnay!

\subsection{Libertarianism}
Libertarianism emphasizes individual freedoms and minimal government intervention. From this perspective Facebooks data practices infringe on individual freedoms by exploiting personal information without consent. Users should have autonomy over their data with corporate actions respecting these individual rights. Ethical evaluations under libertarianism calls for greater transparency and user control over personal information.

\section{Conclusion}
Ethical analysis of Facebooks data practices using five theories highlights some ethical concerns! Each perspective underscores a need for Facebook to realign policies with ethical principles. Findings suggest Facebook must prioritize user privacy to enhance transparency for its mission as a social platform aligns with its operational practices. Recommendations include developing a clear ethical mission statement implementing stricter data protection measures and fostering a corporate culture which values ethical behavior.

%\newpage
\section*{References}

\begin{itemize}
    \item Philosophos.org. (n.d.). Ethical theories: Virtue ethics, utilitarianism, and deontology. Retrieved from \url{https://www.philosophos.org}
    
    \item Libertarianism.org. (n.d.). Consequentialism: A Libertarianism.org guide. Retrieved from \url{https://www.libertarianism.org}
    
    \item Libertarianism.org. (n.d.). A contractarian case for libertarianism. Retrieved from \url{https://www.libertarianism.org}
    
    \item Humanities LibreTexts. (n.d.). 7.6.1: Ethical theories. Retrieved from \url{https://human.libretexts.org}

    \item \url{https://app.originality.ai/share/p80t2fxk76vysza1}
\end{itemize}


\vfill
\section*{Academic Integrity Pledge}
{\color{red}\textit{“This work complies with JMU honor code. I did not give or receive unauthorized help on this assignment.”}}

\end{document}