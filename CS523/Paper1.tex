\documentclass[12pt]{article}
\usepackage{geometry}
\geometry{a4paper, margin=1in}
\usepackage{amsmath}
\usepackage{amsfonts}
\usepackage{amssymb}
\usepackage{graphicx}
\usepackage{hyperref}
\usepackage{xcolor}
 
\title{Plagiarism and Copyright Infringement Laws: Comparison and Contrast}
\author{Abraham J. Reines}
\date{\today}
 
\begin{document}
 
\maketitle
 
\begin{abstract}
This paper compares James Madison University's (JMU) Honor Code with copyright infringement laws. It explores impliccations and consequences of each. Thus, a comprehensive analysis of academic and legal standards is achieved.
\end{abstract}

\section{Introduction}
Plagiarism is misuse of another persons original work, and an academic offense against a university's rules of conduct. Copyright infringement is legal violation of a holder's protections settled in civil court. This paper elucadates:  distinctions and similarities, between Plagiarism and Copyright infringement. As cybersecurity professionals, our field asks for the highest standards of integrity and ethics; consciousness and mindfulness; trustworthiness and impecable values.
 
\section{JMU's Honor Code}
JMU has a rigorous policy on plagiarism. Plagiarism, as defined by JMU's honor code, is using another person's words, ideas, or work without proper attribution. The Honor Code includes:
\begin{itemize}
    \item Proper citation of sources.
    \item Submmission of original work.
    \item Consequences for violations, including academic penalties, suspension, or expulsion.
\end{itemize}
The Honor Code emphasizes academic integrity and respecting intellectual property in both academia and professional workplace. A violation of the Honor Code is a wrong against the University, enforcceable by its rules agreed to by students by virtue of their participationand enrollment. It's an agreement, or covenant, between student and academic entity.
 
\section{Copyright Infringement Laws}
Copyright laws protect rights for a creator, or copyright holder, over original work, with exclusive rights to use, distribute, and profit from unique intellectual and physical creations. Key elements include:
\begin{itemize}
    \item The United States Copyright Act: a legal framework for copyright protection.
    \item The Digital Millennium Copyright Act (DMCA): addrresses copyright issues in the digital age.
    \item Consequences for infringement: statutory damages, fines, and imprisonment.
\end{itemize}
Copyright infringement involves the improper use of copyrighted material, intended to protect the individual creator or copyright holder's original ideas.  Significant legal repercussions are incurred upon violation: it is a wrong against an individual remedied through negotiation or litigation in civil court, however, when theft is involved, usually for financial gain, white collar criminal law may be involved.
 
\section{Comparison}
Plagiarism and copyright infringement share similarities in ethics. Both constructs involve unauthorized use of another's work. They are dishonest practices breaching trust. Penalties can damage reputation and career when plagiarism and copyright infringement alleggations are levied on a student/professional. Proper attribution is required to avoid violations.  In simple terms, strong, compelling evidence is required to make a case for a violation of Plagiarism and/or copyright infringement. In criminal proceedings, proof beyond a reasonable doubt is the controlling standard.
 
\section{Contrast}
Plagiarism and copyright infringement differ in scope:
\begin{itemize}
    \item \textbf{Plagiarism:} An academic violation, usually in the context of educational institutions like JMU. Academic consequences, affecting grades and academic standing.
    \item \textbf{Copyright Infringement:} Governed by federal laws; a legal matter in, for example, the United States of America (USA). Penalties include criminal charges affecting an individual's professional life.
\end{itemize}
 
\section{Case Studies/Examples}
\subsection{Plagiarism in Academia}
Students proven to have plagiarized face academic probation, failure in course in which the plagiarism occurred, or expulsion, depending on severity and institutions policies.
\subsection{Copyright Infringement}
In \textit{Google LLC v. Oracle America Inc.}, Google was sued for using Oracle's copyrighted Java code without permission. This lengthy legal battle had significant financial implications.

\section{Conclusion}
The comparison and contrast of plagiarism and copyright infringement is great for academic and professional integrity. Respecting original creations and attribution for creators are important in both contexts.
 
\begin{thebibliography}{9}
\bibitem{jmu}
James Madison University Honor Code, \url{http://www.jmu.edu/honor/code.shtml}
 
\bibitem{copyright}
U.S. Copyright Office, \url{https://www.copyright.gov/}
 
\bibitem{Plagiarism confidence score URL}
\url{https://app.originality.ai/share/sn7cqtb86gr12u40}
 
\end{thebibliography}
 
\vfill
\section*{Academic Integrity Pledge}
{\color{red}\textit{“This work complies with JMU honor code. I did not give or receive unauthorized help on this assignment.”}}
 
\end{document}