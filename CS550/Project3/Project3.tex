\documentclass{article}
\usepackage[utf8]{inputenc}
\usepackage{listings}
\usepackage{color}
\usepackage{graphicx}
\usepackage{float}
\usepackage{geometry}
\usepackage{xcolor}
\usepackage{booktabs}
\usepackage{longtable}
\usepackage[utf8]{inputenc} % for Unicode input
\usepackage[T1]{fontenc} % for font encoding
\usepackage{lmodern} % Latin Modern font that supports the T1 encoding
\usepackage{textcomp} % for additional symbols
\usepackage{amsmath}
\usepackage{amssymb}
\usepackage{amsthm}
\usepackage{setspace}
\usepackage{enumitem}
\usepackage{hyperref}

% Define colors for code listing
\definecolor{codegreen}{rgb}{0,0.6,0}
\definecolor{codegray}{rgb}{0.5,0.5,0.5}
\definecolor{codepurple}{rgb}{0.58,0,0.82}
\definecolor{backcolour}{rgb}{0.95,0.95,0.92}

% Set listings configuration
\lstset{
  basicstyle=\ttfamily,
  frame=single,
  breaklines=true,
  postbreak=\mbox{\textcolor{red}{$\hookrightarrow$}\space},
  numbers=left,
  numberstyle=\small,
  numbersep=8pt,
  showstringspaces=false,
  tabsize=2,
  language=bash,
  captionpos=b
}

% Adjust margins as needed
\geometry{
 a4paper,
 total={170mm,257mm},
 left=20mm,
 top=20mm,
}

\title{SPY Script Documentation}
\author{Abraham Reines}
\date{\today}

\begin{document}

\maketitle

\section{Script Requirements}

\section{Usage}
The script is invoked with the following syntax:
\begin{lstlisting}[language=bash]
./spy.sh [-t tseconds] [-n count]
\end{lstlisting}
where \texttt{-t tseconds} specifies the time interval between each scan, and \texttt{-n count} specifies the number of times the scan is performed.

\section{Error Checking}
The script includes error checking for invalid options, missing argument values, and non-numeric input for time intervals and count.

\section{Handling Interrupts}
Interrupts, such as Ctrl-C, are gracefully handled by the script to provide a user-friendly exit message.

\section{Script Listing}
\lstinputlisting[language=bash]{spy.sh}

%\newpage
\section{Sample Output Listing}
For the first 10 lines:
\lstinputlisting[firstline=1, lastline=10]{spy_output.txt}

For the last 10 lines:
\lstinputlisting[firstline=2285, lastline=2295]{spy_output.txt}

{\color{red}Full output too long to display in this PDF. The output is intended to be submitted as a .txt file alogn with \texttt{psmonitor.sh} }

\section{Executing the Script}
To execute the script, first ensure it has the appropriate permissions set:
\begin{lstlisting}[language=bash]
chmod +x spy.sh
\end{lstlisting}
Then run the script by providing the desired arguments for time interval and count, e.g.:
\begin{lstlisting}[language=bash]
./spy.sh -t 1 -n 10 > spy_output.txt
\end{lstlisting}
 or use the default values by not providing any arguments.
 
 \section*{References}
\begin{enumerate}
    \item Kili, A. (n.d.). \textit{30 Useful 'ps Command' Examples for Linux Process Monitoring}. Tecmint. Retrieved from \url{https://www.tecmint.com/ps-command-examples-for-linux-process-monitoring/}
    \item Bytexd. (n.d.). \textit{Linux Process Monitoring Using the ps, pstree, top Commands}. Retrieved from \url{https://bytexd.com/linux-process-monitoring/}
    \item Upadhyay, K. (2018, October 6). \textit{How to Monitor and Manage Linux Processes}. Open Source For You. Retrieved from \url{https://www.opensourceforu.com/2018/10/how-to-monitor-and-manage-linux-processes/}
\end{enumerate}

 \vfill
  \section*{Academic Integrity Pledge}
    {\color{red}\textit{“This work complies with the JMU honor code. I did not give or receive unauthorized help on this assignment.”}}

\end{document}
