\documentclass{article}
\usepackage[utf8]{inputenc}
\usepackage{listings}
\usepackage{color}
\usepackage{graphicx}
\usepackage{float}
\usepackage{geometry}
\usepackage{xcolor}
\usepackage{booktabs}
\usepackage{longtable}
\usepackage[utf8]{inputenc} % for Unicode input
\usepackage[T1]{fontenc} % for font encoding
\usepackage{lmodern} % Latin Modern font that supports the T1 encoding
\usepackage{textcomp} % for additional symbols
\usepackage{amsmath}
\usepackage{amssymb}
\usepackage{amsthm}
\usepackage{setspace}
\usepackage{enumitem}

% Define colors for code listing
\definecolor{codegreen}{rgb}{0,0.6,0}
\definecolor{codegray}{rgb}{0.5,0.5,0.5}
\definecolor{codepurple}{rgb}{0.58,0,0.82}
\definecolor{backcolour}{rgb}{0.95,0.95,0.92}

% Style for code
\lstset{
    language=bash,
    backgroundcolor=\color{backcolour},
    commentstyle=\color{codegreen},
    keywordstyle=\color{magenta},
    numberstyle=\tiny\color{codegray},
    stringstyle=\color{codepurple},
    basicstyle=\ttfamily\small,
    breakatwhitespace=false,
    breaklines=true,
    captionpos=b,
    keepspaces=true,
    numbers=left,
    numbersep=5pt,
    showspaces=false,
    showstringspaces=false,
    showtabs=false,
    tabsize=2
}

% Adjust margins as needed
\geometry{
 a4paper,
 total={170mm,257mm},
 left=20mm,
 top=20mm,
}

\title{PSMonitor Script Documentation}
\author{Abraham Reines}
\date{\today}

\begin{document}

\maketitle

\section{Script Requirements}
The script \texttt{psmonitor.sh} is designed to scan the system process table and display all processes running in the system at specified intervals for a specified number of iterations. It takes two optional command-line arguments to customize the time between scans and the number of scans.

\section{Usage}
The script is invoked with the following syntax:
\begin{lstlisting}[language=bash]
./psmonitor.sh [-t tseconds] [-n count]
\end{lstlisting}
where \texttt{-t tseconds} specifies the time interval between each scan, and \texttt{-n count} specifies the number of times the scan is performed.

\section{Sample Output}
A sample output of the script includes the current date and time followed by the list of current processes. The output concludes with an attribution to the script's author and a statement of honor code compliance.

\section{Error Checking}
The script includes error checking for invalid options, missing argument values, and non-numeric input for time intervals and count.

\section{Handling Interrupts}
Interrupts, such as Ctrl-C, are gracefully handled by the script to provide a user-friendly exit message.

\section{Script Listing}
\lstinputlisting[language=bash]{psmonitor.sh}

\section{Executing the Script}
To execute the script, first ensure it has the appropriate permissions set:
\begin{lstlisting}[language=bash]
chmod +x psmonitor.sh
\end{lstlisting}
Then run the script by providing the desired arguments for time interval and count, e.g.:
\begin{lstlisting}[language=bash]
./psmonitor.sh -t 1 -n 10
\end{lstlisting}
 or use the default values by not providing any arguments.

\end{document}
