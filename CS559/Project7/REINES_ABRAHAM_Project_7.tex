\documentclass{article}
\usepackage[utf8]{inputenc}
\usepackage{listings}
\usepackage{color}
\usepackage{graphicx}
\usepackage{float}
\usepackage{geometry}
\usepackage{xcolor}
\usepackage{booktabs}
\usepackage{longtable}
\usepackage[utf8]{inputenc} % for Unicode input
\usepackage[T1]{fontenc} % for font encoding
\usepackage{lmodern} % Latin Modern font that supports the T1 encoding
\usepackage{textcomp} % for additional symbols
\usepackage{amsmath}
\usepackage{amssymb}
\usepackage{amsthm}
\usepackage{setspace}
\usepackage{enumitem}
\usepackage{hyperref}
\usepackage{array} % For table wrapping and advanced features

% Define colors for code listing
\definecolor{codegreen}{rgb}{0,0.6,0}
\definecolor{codegray}{rgb}{0.5,0.5,0.5}
\definecolor{codepurple}{rgb}{0.58,0,0.82}
\definecolor{backcolour}{rgb}{0.95,0.95,0.92}

% Setup the style for the Python code listings
\lstdefinestyle{pythonstyle}{
    language=Python,
    backgroundcolor=\color{backcolour},
    commentstyle=\color{codegreen},
    keywordstyle=\color{magenta},
    numberstyle=\tiny\color{codegray},
    stringstyle=\color{codepurple},
    basicstyle=\ttfamily\small,
    breakatwhitespace=false,
    breaklines=true,
    postbreak=\mbox{\textcolor{red}{$\hookrightarrow$}\space},
    captionpos=b,
    keepspaces=true,
    numbers=left,
    numbersep=5pt,
    showspaces=false,
    showstringspaces=false,
    showtabs=false,
    tabsize=2
}

% Setup the style for plain text listings
\lstdefinestyle{plaintextstyle}{
    language={},
    basicstyle=\ttfamily,
    frame=single,
    breaklines=true,
    postbreak=\mbox{\textcolor{red}{$\hookrightarrow$}\space},
    numbers=left,
    numberstyle=\small,
    numbersep=8pt,
    showstringspaces=false,
    tabsize=2,
    language=bash,
    captionpos=b
}


% Adjust margins as needed
\geometry{
 a4paper,
 total={170mm,257mm},
 left=20mm,
 top=20mm,
}

\title{Algorithm Analysis}
\author{Abraham J. Reines} % Replace [Your Name] with your actual name.
\date{\today} % Use the date of document completion.

\begin{document}

\maketitle

\section*{Results}

{\color{red}Please find all results in the .txt files of the assignment submisstion. }

\section*{Task 7.6.1}

\subsection*{Results}

\begin{table}[htb!]
\centering
\renewcommand{\arraystretch}{1.5} % Adds vertical spacing between rows
\setlength{\tabcolsep}{4pt} % Adjusts the horizontal padding between columns
\begin{tabular}{|>{\centering\arraybackslash}p{7cm}|>{\centering\arraybackslash}p{1.5cm}|>{\centering\arraybackslash}p{1.5cm}|>{\centering\arraybackslash}p{1.5cm}|>{\centering\arraybackslash}p{3cm}|}
\hline
\textbf{Log File Name} & \textbf{Total Events} & \textbf{Error Events} & \textbf{Critical Events} & \textbf{Process IDs Reporting Events} \\
\hline
XUNHUA2023OFFIC\_Setup\_log.evtx & 94 & 0 & 0 & 18412, 12536, \ldots 15492 (Truncated) \\
\hline
DESKTOP-KUHNMLA\_Setup\_log.evtx & 366 & 0 & 0 & 3884, 1824, \ldots 2892 (Truncated) \\
\hline
XUNHUA2023OFFIC\_System\_log.evtx & 29692 & 321 & 8 & 5412, 2780, \ldots 15484 (Truncated) \\
\hline
XUNHUA2023OFFIC\_Security\_log.evtx & 22782 & 0 & 0 & 3596, 3632, \ldots 1416 \\
\hline
DESKTOP-KUHNMLA\_System\_log.evtx & 44357 & 249 & 8 & 5412, 5000, \ldots 9936 (Truncated) \\
\hline
\end{tabular}
\caption{Statistical Analysis of Windows Log Files}
\end{table}

Please note the results are captured by the \texttt{'log\_statistics.txt'} file. This file will be uploaded alogn tside this PDF.

\subsection*{Script}
 
\lstinputlisting[style=pythonstyle, caption={Code to analyze statistics of Windows log files. }]{WindowsLogAnalyzer.py}

\section*{Task 7.6.2}

\subsection*{Results}

\begin{table}[H]
\centering
\renewcommand{\arraystretch}{1.5} % Adds vertical spacing between rows
\setlength{\tabcolsep}{4pt} % Adjusts the horizontal padding between columns
\begin{tabular}{
  |>{\centering\arraybackslash}p{2.5cm}
  |>{\centering\arraybackslash}p{6.5cm}
  |>{\centering\arraybackslash}p{3.5cm}|}
\hline
\textbf{Log File} & \textbf{List of Applications} & \textbf{\# of Events for Each Application} \\
\hline
auth.log & CRON, dbus-daemon, sshd, ...(Truncated) & 3138, 15, 371, ...(Truncated) \\
\hline
kern.log & kernel & 45663 \\
\hline
syslog & CRON, ModemManager, NetworkManager, ...(Truncated) & 1569, 30, 322, ... (Truncated)\\
\hline
\end{tabular}
\caption{Analysis of Ubuntu Log Files}
\end{table}

Please note the results are captured by the \texttt{'UbuntuLoggingResults.txt'} file. This file will be uploaded alogn tside this PDF.

\subsection*{Script}
\lstinputlisting[style=pythonstyle, caption={Code to analyze statistics of Ubuntu log files. }]{UbuntuLogging.py}

\section*{Task 7.6.3}

\subsection*{Results}

\begin{table}[H]
\centering
\renewcommand{\arraystretch}{1.5} % Adds vertical spacing between rows
\setlength{\tabcolsep}{4pt} % Adjusts the horizontal padding between columns
\begin{tabular}{
  |>{\centering\arraybackslash}p{3.5cm}
  |>{\centering\arraybackslash}p{8cm}
  |>{\centering\arraybackslash}p{2.5cm}|}
\hline
\textbf{List of IP Addresses} & \textbf{\# of Events in Each Log File Respectively} & \textbf{Combined \# of Events} \\
\hline
52.167.144.55 & 1, -, - & 5114 \\
40.77.167.15 & 1, -, - & 5114 \\
134.126.120.54 & 5076, 7411, 7933 & 5114, 7504, 7942 \\
... & ... & ... \\
146.190.129.170 & 24, -, 7 & 5114, -, 7942 \\
... & ... & ... \\
\hline
\end{tabular}
\caption{Statistical Analysis of Apache Log Files}
\end{table}

Please note the results are captured by the result files for the Apache analysis. This file will be uploaded alogn tside this PDF.

\subsection*{Script}
\lstinputlisting[style=pythonstyle, caption={Code to analyze statistics of Apache log files. }]{ApacheLogging.py}

\section*{References}

\begin{enumerate}
  \item Real Python. (n.d.). \textit{Speed Up Your Python Program With Concurrency}. Retrieved from https://realpython.com
  \item Pascariu, C. (2022, September 13). \textit{Log File Analysis with Python}. Pluralsight. Retrieved from 
  https://www.pluralsight.com
  \item Toptal®. (n.d.). \textit{Python Multithreading Tutorial: Concurrency and Parallelism}. Retrieved from 
  https://www.toptal.com
  \item HuangYiwei. (2019, December 30). \textit{concurrent-log}. PyPI. Retrieved from https://pypi.org/project/concurrent-log/
\end{enumerate}


\vfill 
  \section*{Academic Integrity Pledge}
   {\color{red}\textit{“This work complies with the JMU honor code. I did not give or receive unauthorized help on this assignment.”}}
\end{document}
