\documentclass{article}
\usepackage{graphicx}
\usepackage{float}
\usepackage{geometry}
\usepackage{xcolor}
\usepackage{booktabs}
\usepackage{longtable}
\usepackage[utf8]{inputenc} % for Unicode input
\usepackage[T1]{fontenc} % for font encoding
\usepackage{lmodern} % Latin Modern font that supports the T1 encoding
\usepackage{textcomp} % for additional symbols
\usepackage{amsmath}
\usepackage{amssymb}
\usepackage{amsthm}
\usepackage{setspace}
\usepackage{enumitem}
\usepackage{listings} % For code listing

% Custom colors for code highlighting
\definecolor{codegreen}{rgb}{0,0.6,0}
\definecolor{codegray}{rgb}{0.5,0.5,0.5}
\definecolor{codepurple}{rgb}{0.58,0,0.82}
\definecolor{backcolour}{rgb}{0.95,0.95,0.92}

% Set listings configuration
\lstset{
  basicstyle=\ttfamily,
  frame=single,
  breaklines=true,
  postbreak=\mbox{\textcolor{red}{$\hookrightarrow$}\space},
  numbers=left,
  numberstyle=\small,
  numbersep=8pt,
  showstringspaces=false,
  tabsize=2,
  language=bash,
  captionpos=b
}

% Adjust margins as needed
\geometry{
 a4paper,
 total={170mm,257mm},
 left=20mm,
 top=20mm,
}

\title{Databases and Database Security}
\author{Abraham J. Reines}
\date{\today}

\begin{document}

\maketitle

\section*{Introduction}

\section*{4.6.1 Install MySQL to Ubuntu machine}

\subsection*{Output of Requirements of 4.6.1:}
\begin{figure}[H]
    \centering
    \includegraphics[width=12cm]{Screenshot 2024-02-26 at 10.19.07 AM.png}
    \caption{}
\end{figure}
\lstinputlisting[firstline=1, lastline=13,caption={Requirements for 4.6.1}]{Output_For_4.6.1.txt}

To install MySQL on your Ubuntu Virtual Machine (VM), these steps were taken:

\begin{enumerate}
    \item Open Ubuntu terminal.
    \item Update package index using the command: \texttt{sudo apt-get update}.
    \item Install MySQL with the command: \texttt{sudo apt-get install mysql-server}.
    \item Secure your MySQL installation with: \texttt{sudo mysql\_secure\_installation}.
    \item Connect to the MySQL database with the command: \texttt{mysql -u root -p}.
\end{enumerate}

After successfully logging into MySQL, execute the following query to retrieve the list of users:

\begin{verbatim}
SELECT user FROM mysql.user;
\end{verbatim}

A screenshot is included above of the terminal showing both the executed command and the resulting output. 

\section*{4.6.2 Create a database, two tables, one view, and insert data to them}

\subsection*{Output of Requirements for 4.6.2:}

\begin{figure}[H]
    \centering
    \includegraphics[width=10cm]{Screenshot 2024-03-03 at 9.12.22 AM.png}
    \caption{A new database named  \texttt{cs559dbsec} was created.}
\end{figure}

\begin{figure}[H]
    \centering
    \includegraphics[width=\linewidth]{Screenshot 2024-02-27 at 8.40.57 AM.png}
    \caption{Within the \texttt{cs559dbsec} database, a table named \texttt{customers} was created with attributes as per Section 4.5.3.}
\end{figure}

\begin{figure}[H]
    \centering
    \includegraphics[width=\linewidth]{Screenshot 2024-02-27 at 8.40.47 AM.png}
    \caption{Similarly, a table named \texttt{queries} was established within the \texttt{cs559dbsec} database with attributes outlined in Section 4.5.3.}
\end{figure}

\begin{figure}[H]
    \centering
    \includegraphics[width=\linewidth]{Screenshot 2024-02-27 at 8.40.40 AM.png}
    \caption{A view called \texttt{restrictedcustomers} was created within the \texttt{cs559dbsec} database to encapsulate certain attributes specified in Section 4.5.3.}
\end{figure}

\begin{figure}[H]
    \centering
    \includegraphics[width=\linewidth]{Screenshot 2024-02-27 at 10.20.00 AM.png}
    \caption{Data was inserted into the \texttt{customers} table from the specified source.}
\end{figure}

\begin{figure}[H]
    \centering
    \includegraphics[width=\linewidth]{Screenshot 2024-02-27 at 10.20.21 AM.png}
    \caption{Data was inserted into the \texttt{queries} table from the provided resource.}
\end{figure}

%\lstinputlisting[firstline=1, lastline=13,caption={Requirements for 4.6.1}]{Output_For_4.6.2.txt}

\subsection*{Database and Table Creation with Data Insertion}

The following steps were executed to set up the MySQL database environment:

\begin{enumerate}
    \item \textbf{Database Creation:} A new database named \texttt{cs559dbsec} was created.
    
    \item \textbf{Table Creation - Customers:} Within the \texttt{cs559dbsec} database, a table named \texttt{customers} was created with attributes as per Section 4.5.3.
    
    \item \textbf{Table Creation - Queries:} Similarly, a table named \texttt{queries} was established within the \texttt{cs559dbsec} database with attributes outlined in Section 4.5.3.
    
    \item \textbf{View Creation - Restricted Customers:} A view called \texttt{restrictedcustomers} was created within the \texttt{cs559dbsec} database to encapsulate certain attributes specified in Section 4.5.3.
    
    \item \textbf{Data Insertion - Customers:} Data was inserted into the \texttt{customers} table from the specified source.
    
    \item \textbf{Data Insertion - Queries:} Data was inserted into the \texttt{queries} table from the provided resource.
\end{enumerate}

For each of the above actions, a screenshot displaying both the command input and the successful result was captured and included in this report. The screenshots serve as a validation of the correctly executed commands and the successful outcomes.

\begin{enumerate}
    \item Create a new database:
    \begin{verbatim}
    CREATE DATABASE cs559dbsec;
    \end{verbatim}

    \item Select the database for use:
    \begin{verbatim}
    USE cs559dbsec;
    \end{verbatim}

    \item Create a table named \texttt{customers}:
    \begin{verbatim}
    CREATE TABLE `customers` (
        `loginName` varchar(20) NOT NULL,
        `password` char(255) NOT NULL,
        `lastName` varchar(50) NOT NULL,
        `firstName` varchar(50) NOT NULL,
        `middleName` char(30) DEFAULT NULL,
        `jmuEID` bigint(20) unsigned NOT NULL AUTO_INCREMENT,
        `ssn` int(9) unsigned DEFAULT NULL,
        `studentID` int(9) unsigned DEFAULT NULL,
        `creditCardNumber` int(16) unsigned DEFAULT NULL,
        `nameOnCard` varchar(50) DEFAULT NULL,
        `cardExpirationDate` date DEFAULT NULL,
        `emailAddress` varchar(250) DEFAULT NULL,
        `createTime` datetime DEFAULT NULL,
        `street` varchar(50) DEFAULT NULL,
        `city` varchar(50) DEFAULT NULL,
        `state` char(2) DEFAULT NULL,
        `zip` char(10) DEFAULT NULL,
        PRIMARY KEY (`loginName`),
        UNIQUE KEY `jmuEID` (`jmuEID`)
    ) ENGINE=MyISAM AUTO_INCREMENT=107723521 DEFAULT CHARSET=latin1;
    \end{verbatim}

    \item Create a table named \texttt{queries}:
    \begin{verbatim}
    CREATE TABLE `queries` (
        `queryName` varchar(250) NOT NULL,
        `queryTime` datetime NOT NULL,
        `queryIP` varchar(15) NOT NULL,
        PRIMARY KEY (`queryName`, `queryTime`)
    ) ENGINE=MyISAM DEFAULT CHARSET=latin1;
    \end{verbatim}

    \item Create a view named \texttt{restrictedcustomers}:
    \begin{verbatim}
    CREATE VIEW `restrictedcustomers` AS SELECT
        `loginName`, `lastName`, `firstName`, `middleName`,
        `emailAddress`, `street`, `city`, `state`, `zip`
    FROM `customers`;
    \end{verbatim}

    \item Insert data into the \texttt{customers} table:
    \begin{verbatim}
    INSERT INTO `customers` VALUES
    ('addiek','password','Addie','Kyle','William',...),
    ('alenrm','password','Allen','Rafael','Mark',...),
    -- additional rows of data
    \end{verbatim}

    \item Insert data into the \texttt{queries} table:
    \begin{verbatim}
    INSERT INTO `queries` VALUES
    ('wangxx','2012-09-16 22:12:03','192.168.101.110'),
    ('x','2012-09-19 14:34:01','192.168.101.154'),
    -- additional rows of data
    \end{verbatim}
\end{enumerate}

Each step was documented with a screenshot showing both the command and the resulting output to ensure proper verification and to meet the assignment criteria.


\section*{4.6.3 Fix password storage}

\begin{figure}[H]
    \centering
    \includegraphics[width=\linewidth]{Screenshot 2024-03-03 at 8.51.45 AM.png}
    \caption{Using HashingForDB.py; Printing all hashed passwords to verify the process and ensure all passwords are hashed.}
\end{figure}

\lstinputlisting[caption={Script for 4.6.3}]{HashingForDatabase.py}

\subsection*{Objective}
The goal is to alter the schema of the \textit{customers} table to store passwords securely without necessitating the creation of new tables or migration of existing data.

\subsection*{Methodology}
The solution involves implementing a hashing algorithm to store secure password hashes instead of plaintext passwords. The \texttt{bcrypt} hashing function was selected for its strong security features, including salted hashes and resistance to brute-force search attacks.

\subsection{Execution}
The execution of securing the passwords in the database involved several key steps:
\begin{enumerate}
    \item Establishing a secure connection to the database with \texttt{mysql.connector}.
    \item Adding a new column \texttt{password\_hash} to the \textit{customers} table to store the hashed passwords, if it does not already exist.
    \item Retrieving all customer records which have a null or empty \texttt{password\_hash}.
    \item For each customer, the plaintext password is encoded and hashed using \texttt{bcrypt} with a generated salt.
    \item The database is then updated with the new hashed passwords for each customer.
    \item Committing the changes to the database to ensure data integrity.
    \item Printing all hashed passwords to verify the process and ensure all passwords are hashed.
    \item Closing the database connection securely.
\end{enumerate}
This method ensured all plaintext passwords were replaced with secure hashes without affecting the existing data structure or requiring data migration.

\subsection*{Testing and Verification}
The modified table schema and the password update process were tested on a local machine. Hashed passwords were verified to ensure they matched the original plaintext when hashed.

\subsection*{Results}
The \textit{customers} table now stores password hashes instead of plaintext passwords. Sample output after hashing is shown below, and also above in Figure 8:

\lstinputlisting[caption={Hashes for 4.6.3}]{Output_For_4.6.3.txt}

\subsection*{Conclusion}
The \textit{customers} table has been successfully updated to use hashed passwords, significantly improving the security of the stored user credentials. This change has been implemented without disrupting existing data structures or requiring data migration, aligning with the best practices for secure password storage.

\section*{4.6.4 Create three new users and assign privileges}

\begin{figure}[H]
    \centering
    \includegraphics[width=\linewidth]{Screenshot 2024-02-27 at 10.06.49 AM.png}
    \caption{This user is intended for back-end operations with full privileges on the \texttt{cs559dbsec} database.}
\end{figure}

\begin{figure}[H]
    \centering
    \includegraphics[width=\linewidth]{Screenshot 2024-02-29 at 4.32.46 PM.png}
    \caption{This user has all privileges to all databases and the ability to grant privileges to other users, suitable for database administrators.}
\end{figure}

\begin{figure}[H]
    \centering
    \includegraphics[width=\linewidth]{Screenshot 2024-02-29 at 4.31.50 PM.png}
    \caption{This user is intended for back-end operations with full privileges on the \texttt{cs559dbsec} database.}
\end{figure}

\begin{figure}[H]
    \centering
    \includegraphics[width=\linewidth]{Screenshot 2024-02-29 at 4.33.25 PM.png}
    \caption{This user is designed for web applications, with read-only access to the \texttt{restrictedcustomers} view and write-only access to the \texttt{queries} table.}
\end{figure}

\begin{figure}[H]
    \centering
    \includegraphics[width=\linewidth]{Screenshot 2024-02-29 at 5.01.31 PM.png}
    \caption{The implemented solution for securing the password storage in the \textit{customers} table is in full compliance with the requirements.}
\end{figure}

\lstinputlisting[caption={Script for 4.6.4}]{PrivilegesForDB.py}

To facilitate different levels of interaction with the databases, the following users were created, each with specific privileges:

\begin{enumerate}
    \item \textbf{User Creation - backenduser:}
    This user is intended for back-end operations with full privileges on the \texttt{cs559dbsec} database.
    \begin{verbatim}
    CREATE USER 'backenduser'@'localhost' IDENTIFIED BY 'password';
    GRANT ALL PRIVILEGES ON cs559dbsec.* TO 'backenduser'@'localhost';
    FLUSH PRIVILEGES;
    \end{verbatim}

    \item \textbf{User Creation - webuser:}
    This user is designed for web applications, with read-only access to the \texttt{restrictedcustomers} view and write-only access to the \texttt{queries} table.
    \begin{verbatim}
    CREATE USER 'webuser'@'localhost' IDENTIFIED BY 'password';
    GRANT SELECT ON cs559dbsec.restrictedcustomers TO 'webuser'@'localhost';
    GRANT INSERT ON cs559dbsec.queries TO 'webuser'@'localhost';
    FLUSH PRIVILEGES;
    \end{verbatim}

    \item \textbf{User Creation - dbmanager:}
    This user has all privileges to all databases and the ability to grant privileges to other users, suitable for database administrators.
    \begin{verbatim}
    CREATE USER 'dbmanager'@'localhost' IDENTIFIED BY 'password';
    GRANT ALL PRIVILEGES ON *.* TO 'dbmanager'@'localhost' WITH GRANT OPTION;
    FLUSH PRIVILEGES;
    \end{verbatim}
\end{enumerate}

For each of the user creations and privilege assignments, a screenshot capturing both the command and the successful execution result was taken and is included in the report. Additionally, a python script was written to improve the efficiency of this process.

\subsection*{Script Compliance with Requirements}
The implemented solution for securing the password storage in the \textit{customers} table is in full compliance with the prescribed requirements:
\begin{itemize}
    \item The table schema was modified to include a column for storing password hashes. This was accomplished without creating a new table or migrating existing data, satisfying the stipulation for in-place modification.
    \item A secure, contemporary hashing algorithm was applied to all existing passwords, converting them into non-reversible hashes. This process was performed using a Python script, ensuring  the procedure is repeatable and consistent.
    \item The solution is generic in nature, allowing all data to be updated with a single execution of the script. This fulfills the requirement for a universal update mechanism.
    \item The procedure was thoroughly tested on a local machine, ensuring the solution is both effective and secure before deployment.
\end{itemize}
The methodical approach adopted in this solution prioritizes security without compromising on the database's performance or integrity. 

\section*{4.6.5 Examine a separate SQLite database}

\begin{figure}[H]
    \centering
    \includegraphics[width=\linewidth]{Screenshot 2024-03-02 at 9.30.05 AM.png}
    \caption{The SQLite database provided by the National Security Agency (NSA) was queried using a Python script, \texttt{NSATime.py}, to identify any records which correspond to the USCG's signal data based on specific geographic and temporal parameters. The database \texttt{database.db} and the USCG log \texttt{USCG.log} were utilized as inputs.}
\end{figure}

\subsection*{Script listing}
\lstinputlisting[caption={Script for 4.6.5}]{NSATime.py}
\lstinputlisting[caption={Results for 4.6.5}]{Output_For_4.6.5.txt}

This section details the process and results of examining a separate SQLite database to aid the US Coast Guard (USCG) in identifying signals similar to those unregistered ones recorded over 30 nautical miles away from the continental US.

\subsection*{Methodology}
The SQLite database provided by the National Security Agency (NSA) was queried using a Python script, \texttt{NSATime.py}, to identify any records which correspond to the USCG's signal data based on specific geographic and temporal parameters. The database \texttt{database.db} and the USCG log \texttt{USCG.log} were utilized as inputs.

\subsection*{Results}
The script successfully identified records with geographic coordinates within 1/100th of a degree of the signal origin and timestamps within 10 minutes of the recorded signal. The extracted record IDs and relevant details were as follows:

\begin{itemize}
    \item Record ID: 179
    \item Coordinates: Latitude 29.65676, Longitude -87.77498
    \item Timestamp: 02/06/2023 03:35:40
    \item ... and other details ...
\end{itemize}

\begin{itemize}
    \item Record ID: 539
    \item Coordinates: Latitude 29.65108, Longitude -87.77264
    \item Timestamp: 02/06/2023 03:40:06
    \item ... and other details ...
\end{itemize}

The full list of identified events and their details can be found in the attached \texttt{Output\_For\_4.6.5.txt}.

\section*{4.6.6 Bonus task: Database encryption}

\lstinputlisting[caption={Script for 4.6.5}]{DBEncryptor.py}

\begin{figure}[H]
    \centering
    \includegraphics[width=\linewidth]{Screenshot 2024-03-02 at 4.44.43 PM.png}
    \caption{To encrypt the specified columns, a Python script \texttt{DBEncryptor.py} was utilized. This script executes a series of SQL commands to modify the existing \textit{customers} table, enabling encryption on the \textit{ssn} and \textit{creditCardNumber} fields without the need to migrate data or create new tables.}
\end{figure}

%\lstinputlisting[caption={Results for 4.6.5}]{Output_For_4.6.5.txt}

\subsection*{Methodology}
To encrypt the specified columns, a Python script \texttt{DBEncryptor.py} was utilized. This script executes a series of SQL commands to modify the existing \textit{customers} table, enabling encryption on the \textit{ssn} and \textit{creditCardNumber} fields without the need to migrate data or create new tables.

\subsection*{Implementation}
The script uses the \texttt{cryptography} library in Python to perform the encryption and decryption operations, ensuring the encryption is consistent and reversible for legitimate use while maintaining the integrity and functionality of the database.

\subsection*{Execution}
The script was executed on an Ubuntu system with the following steps:
\begin{enumerate}
    \item Establishing a connection to the MySQL database.
    \item Encrypting data within the \textit{ssn} and \textit{creditCardNumber} columns.
    \item Updating the table with encrypted data.
\end{enumerate}

\subsection*{Results}
The execution of the script resulted in the successful encryption of the \textit{ssn} and \textit{creditCardNumber} columns. Sample output from the script is presented below:

\begin{lstlisting}[language=SQL]
LoginName: adviewk, Encrypted SSN: b'AAAAB3NzaC1yc2EAAAADAQABAAABAQC...', 
Encrypted Credit Card: b'AAAAB3NzaC1yc2EAAAABIwAAAQEA...'
...
Encryption of data completed and printed.
\end{lstlisting}

\subsection*{Security Justification}
The employed encryption methodology ensures sensitive customer data is secured within the database, rendering it unreadable without the proper decryption key. This approach mitigates the risk of data breaches and unauthorized access.

\subsection*{Testing and Verification}
To verify the encryption, select queries were executed to ensure the data, when decrypted, matched the original plaintext information. The script was tested on the local machine, and the entire process was documented through screenshots captured after each command execution.

\subsection*{Conclusion}
The encryption of the \textit{ssn} and \textit{creditCardNumber} columns was achieved without affecting the existing database structure. 

\vfill
  \section*{Academic Integrity Pledge}
    {\color{red}\textit{“This work complies with the JMU honor code. I did not give or receive unauthorized help on this assignment.”}}

\end{document}