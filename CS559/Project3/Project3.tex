\documentclass{article}
\usepackage{graphicx}
\usepackage{float}
\usepackage{geometry}
\usepackage{xcolor}
\usepackage{booktabs}
\usepackage{longtable}
\usepackage[utf8]{inputenc} % for Unicode input
\usepackage[T1]{fontenc} % for font encoding
\usepackage{lmodern} % Latin Modern font that supports the T1 encoding
\usepackage{textcomp} % for additional symbols
\usepackage{amsmath}
\usepackage{amssymb}
\usepackage{amsthm}
\usepackage{setspace}
\usepackage{enumitem}
\usepackage{listings} % For code listing

% Custom colors for code highlighting
\definecolor{codegreen}{rgb}{0,0.6,0}
\definecolor{codegray}{rgb}{0.5,0.5,0.5}
\definecolor{codepurple}{rgb}{0.58,0,0.82}
\definecolor{backcolour}{rgb}{0.95,0.95,0.92}

% Style for code
\lstset{
    language=Python,
    backgroundcolor=\color{backcolour},
    commentstyle=\color{codegreen},
    keywordstyle=\color{magenta},
    numberstyle=\tiny\color{codegray},
    stringstyle=\color{codepurple},
    basicstyle=\ttfamily\small,
    breakatwhitespace=false,
    breaklines=true,
    captionpos=b,
    keepspaces=true,
    numbers=left,
    numbersep=5pt,
    showspaces=false,
    showstringspaces=false,
    showtabs=false,
    tabsize=2
}

% Adjust margins as needed
\geometry{
 a4paper,
 total={170mm,257mm},
 left=20mm,
 top=20mm,
}

\title{Linux File Permissions}
\author{Abraham J. Reines}
\date{\today}

\begin{document}

\maketitle

\section*{Understanding Linux file permissions}

\begin{enumerate}
    \item \textbf{Can Alice read Bob's file \texttt{data.txt}? Why?}\\
    Yes, the file \texttt{data.txt} permissions are set as \texttt{-rw-r----r--}, which grants read access to all users including Alice.
    
    \item \textbf{Can Alice remove Bob's file \texttt{data.txt}? Why?}\\
    No, Alice lacks write permission on Bob's home directory \texttt{/cs/home/stu/bob}, which is required to delete files.
    
    \item \textbf{Can Alice read Bob's file \texttt{secret.txt}? Why?}\\
    No, the file \texttt{secret.txt} has permissions \texttt{-rw-r-----}, which restricts read access to the owner and group `faculty`. Since Alice is not a member of the `faculty` group, she does not have read access.
    
    \item \textbf{Can Alice remove Bob's file \texttt{secret.txt}? Why?}\\
    No, Alice cannot delete \texttt{secret.txt} due to the absence of write permission in Bob's home directory.
    
    \item \textbf{When Bob creates a new file with the command \texttt{echo “My Super Secret is b7d5d78shes” > mysecret.txt}, what are the full permissions of this file?}\\
    The new file \texttt{mysecret.txt} will inherit the `setgid` bit of the directory and receive the permissions \texttt{-rw-r-----}, assuming the `umask` is set to \texttt{0022} which is typical for user files.
\end{enumerate}

\section*{Setting Linux file permissions}

\begin{enumerate}
    \item \textbf{Can Bob change the permissions so that all other students in \texttt{csmajor} can read \texttt{data.txt}, but any other users who are not in \texttt{csmajor} cannot?}\\
    Yes, by executing \texttt{chmod 640 /cs/home/stu/bob/data.txt}, Bob sets the file \texttt{data.txt} to be readable and writable by the owner and only readable by the group.
    
    \item \textbf{If Bob wants to set the default permission of his new files to be readable and writable by himself and the group, and readable by others, what commands should he use? Hint: use umask.}\\
    Bob should issue the command \texttt{umask 002} to ensure files are created with \texttt{rw-rw-r--} permissions and directories with \texttt{rwxrwxr-x}.
\end{enumerate}

\section*{A more complex case}

Alice should execute the following commands to configure \texttt{treasure.txt} permissions appropriately:

\begin{lstlisting}[language=bash]
# Assume Alice has the necessary permissions to perform these actions
# Create a group named 'treasure_group' and add Bob to it
sudo groupadd treasure_group
sudo usermod -aG treasure_group bob

# Change the group ownership of treasure.txt to 'treasure_group'
chgrp treasure_group treasure.txt

# Set permissions for the file to read and write for owner and group
chmod 660 treasure.txt

# Set an ACL for Charlie to read but not write
setfacl -m u:charlie:r-- treasure.txt

# Remove all permissions for others
setfacl -m o::--- treasure.txt

# Verify permissions
getfacl treasure.txt
\end{lstlisting}

Alice should verify these settings in her environment to ensure they are correct. Should she face permission-related issues, the system administrator’s intervention may be required.

\vfill
  \section*{Academic Integrity Pledge}
    {\color{red}\textit{“This work complies with the JMU honor code. I did not give or receive unauthorized help on this assignment.”}}

\end{document}