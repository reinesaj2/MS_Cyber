\documentclass{article}
\usepackage{listings}
\usepackage{xcolor}
\usepackage{changepage}

% Custom colors for code highlighting
\definecolor{codegreen}{rgb}{0,0.6,0}
\definecolor{codegray}{rgb}{0.5,0.5,0.5}
\definecolor{codepurple}{rgb}{0.58,0,0.82}
\definecolor{backcolour}{rgb}{0.95,0.95,0.92}

% Style for Python code
\lstset{
    language=Python,
    backgroundcolor=\color{backcolour},
    commentstyle=\color{codegreen},
    keywordstyle=\color{magenta},
    numberstyle=\tiny\color{codegray},
    stringstyle=\color{codepurple},
    basicstyle=\ttfamily\small,
    breakatwhitespace=false,
    breaklines=true,
    captionpos=b,
    keepspaces=true,
    numbers=left,
    numbersep=5pt,
    showspaces=false,
    showstringspaces=false,
    showtabs=false,
    tabsize=2
}

\begin{document}

\title{Project 5: Pig Latin Translator}
\author{Abraham Reines}
\date{\today}

\maketitle

\section{Description}
This script implements a Pig Latin translator. The goal is to translate an English sentence into Pig Latin. Pig Latin is a language game where words are altered by a set of rules. If a word starts with a vowel, "yay" is added to the end. If a word starts with a consonant, all the consonants before the first vowel are moved to the end of the word and "ay" is added. The script uses a function called \texttt{pig\_latin} to perform the translation for a single word. It then uses another function called \texttt{translate\_to\_pig\_latin} to translate an entire sentence by splitting it into words and applying the \texttt{pig\_latin} function to each word. The translated sentence is printed as the output. The script utilizes string manipulation, loops, conditionals, and functions.

\section{Code}
\begin{lstlisting}
#!/usr/bin/env python3
# -*- coding: utf-8 -*-
"""
Created on Thu Jun 22 08:46:21 2023
Modified on Wed Jun 22 10:00:00 2023
@author: abrahamreines
"""

def pig_latin(word):
    vowels = 'aeiou'
    if word[0] in vowels:
        return word + 'yay'
    elif word[0] == 'y':
        return word[1:] + word[0] + 'ay'
    else:
        consonant_cluster = ''
        for letter in word:
            if letter not in vowels:
                consonant_cluster += letter
            else:
                break
        return word[len(consonant_cluster):] + consonant_cluster + 'ay'

def translate_to_pig_latin(sentence):
    words = sentence.split()
    pig_latin_words = [pig_latin(word.lower()) for word in words]
    return ' '.join(pig_latin_words)

# Ask the user for input
user_input = input("Enter an English sentence: ")
pig_latin_sentence = translate_to_pig_latin(user_input)
print("Pig Latin translation:", pig_latin_sentence)
\end{lstlisting}

\section{Example Output}

\begin{adjustwidth}{-4cm}{}\begin{verbatim}
Enter an English sentence: write a program named pig.py that translates a string input by the user from english
to pig latin
Pig Latin translation: itewray ayay ogrampray amednay ig.pypay atthay anslatestray ayay ingstray inputyay byay
ethay useryay omfray englishyay otay igpay atinlay
\end{verbatim}\end{adjustwidth}

\section{Code Explanation}
The script defines two functions: \texttt{pig\_latin} and \texttt{translate\_to\_pig\_latin}. The \texttt{pig\_latin} function takes a word as input and applies the rules of Pig Latin translation to convert it. It checks if the word starts with a vowel, and if so, adds "yay" to the end. If the word starts with a consonant or "y", it moves the consonant cluster or "y" to the end and adds "ay". The \texttt{translate\_to\_pig\_latin} function takes a sentence as input, splits it into words, applies the \texttt{pig\_latin} function to each word, and joins the translated words back into a sentence.

The script then prompts the user to enter an English sentence and calls the \texttt{translate\_to\_pig\_latin} function to translate it. The translated sentence is printed as the output.

The script uses string manipulation, loops, conditionals, and functions to implement the translation logic and provide a user-friendly interaction.

\end{document}