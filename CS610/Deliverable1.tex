\documentclass{article}
\usepackage[utf8]{inputenc}
\usepackage{listings}
\usepackage{color}
\usepackage{graphicx}
\usepackage{float}
\usepackage{geometry}
\usepackage{xcolor}
\usepackage{booktabs}
\usepackage{longtable}
\usepackage[utf8]{inputenc} % for Unicode input
\usepackage[T1]{fontenc} % for font encoding
\usepackage{lmodern} % Latin Modern font that supports the T1 encoding
\usepackage{textcomp} % for additional symbols
\usepackage{amsmath}
\usepackage{amssymb}
\usepackage{amsthm}
\usepackage{setspace}
\usepackage{enumitem}
\usepackage{hyperref}
\usepackage{array} % For table wrapping and advanced features

% Define colors for code listing
\definecolor{codegreen}{rgb}{0,0.6,0}
\definecolor{codegray}{rgb}{0.5,0.5,0.5}
\definecolor{codepurple}{rgb}{0.58,0,0.82}
\definecolor{backcolour}{rgb}{0.95,0.95,0.92}

% Setup the style for the Python code listings
\lstdefinestyle{pythonstyle}{
    language=Python,
    backgroundcolor=\color{backcolour},
    commentstyle=\color{codegreen},
    keywordstyle=\color{magenta},
    numberstyle=\tiny\color{codegray},
    stringstyle=\color{codepurple},
    basicstyle=\ttfamily\small,
    breakatwhitespace=false,
    breaklines=true,
    postbreak=\mbox{\textcolor{red}{$\hookrightarrow$}\space},
    captionpos=b,
    keepspaces=true,
    numbers=left,
    numbersep=5pt,
    showspaces=false,
    showstringspaces=false,
    showtabs=false,
    tabsize=2
}

% Setup the style for plain text listings
\lstdefinestyle{plaintextstyle}{
    language={},
    basicstyle=\ttfamily,
    frame=single,
    breaklines=true,
    postbreak=\mbox{\textcolor{red}{$\hookrightarrow$}\space},
    numbers=left,
    numberstyle=\small,
    numbersep=8pt,
    showstringspaces=false,
    tabsize=2,
    language=bash,
    captionpos=b
}

\title{CS610 - Project 1: QKD-Encrypted LLM}
\author{Abraham J. Reines}
\date{\today}

\begin{document}

\maketitle

\tableofcontents

\section{Introduction}
This project deploys a small pre-trained language model (GPT-2) on the `stu.cs.jmu.edu` server. It also uses a Quantum Key Distribution (QKD) to encrypt communications between any client device and server. The goal is to explore quantum cryptographic techniques and their application in modern systems for secure communications, aligning with the course's focus on networking and security.

\section{High-Level Design}

\subsection{System Architecture}
The system is designed as a client-server architecture:
\begin{itemize}
    \item \textbf{Server Side:} The server hosts the pre-trained LLM (GPT-2) and handles encrypted communication with clients. It runs on `stu.cs.jmu.edu`.
    \item \textbf{Client Side:} Clients communicate with the server over a secure network. They use QKD to exchange encryption keys, which are used to encrypt and decrypt communications.
\end{itemize}

\subsection{Quantum Key Distribution (QKD)}
A QKD module is implemented to securely exchange encryption keys between the client and server.

\subsection{Encryption and Decryption}
\begin{itemize}
    \item The LLM model is stored on the server and is accessed by clients via encrypted communications.
    \item When a client connects, the QKD process is initiated to securely exchange the encryption key.
    \item The client uses the key to decrypt the LLM model's responses, and all subsequent communications are encrypted with this key.
\end{itemize}

\section{Implementation Details}

\subsection{Programming Language}
The project is implemented in Python, which is supported on `stu.cs.jmu.edu`. Python is chosen for its machine learning and cryptography libraries.

\subsection{Specialized Libraries}
\begin{itemize}
    \item \textbf{Hugging Face Transformers:} Used to deploy the GPT-2 model.
    \item \textbf{PyCryptodome:} Used for encryption and decryption processes.
\end{itemize}

\subsection{Key Modules}
\begin{itemize}
    \item \textbf{server.py:} Manages the LLM. Handles client connections.
    \item \textbf{client.py:} Initiates connections with the server, handles QKD, and decrypts the LLM model responses.
    \item \textbf{model.py:} A GPT-2 module is implemented as a stand-in for more sophisticated/larger models such as Ollama. This component is intended to be "proof of concept".
    \item \textbf{qkd.py:} Implements the QKD simulation, ensuring key exchange between client and server.
\end{itemize}

\lstinputlisting[style=pythonstyle, caption={QKD Module}]{LLM-QKD_project/qkd.py}
\lstinputlisting[style=pythonstyle, caption={LLM Module}]{LLM-QKD_project/model.py}
\lstinputlisting[style=pythonstyle, caption={Client Module}]{LLM-QKD_project/client.py}
\lstinputlisting[style=pythonstyle, caption={Server Module}]{LLM-QKD_project/server.py}

\section{Compiling and Running the Project}

\subsection{Setup Instructions}
\begin{itemize}
    \item Ensure Python 3 is installed on both the server and client machines.
    \item Set up a virtual environment:
    \begin{verbatim}
    python3 -m venv QKD
    source QKD/bin/activate
    \end{verbatim}
    \item Install the required libraries using pip:
    \begin{verbatim}
    pip install -r requirements.txt
    \end{verbatim}
    \item Place the `server.py`, `client.py`, and `qkd.py` files in the appropriate directories on the server and client machines.
\end{itemize}

\subsection{Running the Server}

Run on 'stu' server:
\begin{verbatim}
python3 server.py
\end{verbatim}

\subsection{Running the Client}

Run on client system:
\begin{verbatim}
python3 client.py
\end{verbatim}

\section{Known Issues}
\begin{itemize}
    \item Next steps will be to implement the QKD explicitly, potentially leveraging the 'Qiskit' python library. This remains untested. 
    \item The model response decryption process may introduce some latency, depending on the size of the LLM and the computational power of the client machine.
\end{itemize}

\section{Conclusion}
This project successfully integrates quantum cryptographic techniques with machine learning, offering a secure approach to deploying LLMs in a networked environment. By incorporating Quantum Key Distribution (QKD), the system is fortified against emerging quantum computing threats, ensuring the confidentiality and integrity of data even in advanced threat landscapes. The implications of this project extend beyond academic exploration; it introduces the potential for highly secure, custom-trained LLM responses leveraged by Special Operations Forces (SOF) and U.S. government agencies. This application could play a critical role in enhancing secure communications, intelligence analysis, and decision-making processes in sensitive and mission-critical operations, with robust and secure functionality. 

\end{document}