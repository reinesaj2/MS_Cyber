\documentclass[12pt]{article}

% Packages
\usepackage{amsmath}
\usepackage{amssymb}
\usepackage{amsthm}
\usepackage{geometry}
\usepackage{setspace}
\usepackage{xcolor}
\usepackage{enumitem}

% Define theorem, definition, and proof environments
\newtheorem{theorem}{Theorem}
\newtheorem{definition}{Definition}
\newtheorem{lemma}{Lemma}
\newtheorem{corollary}{Corollary}
\newtheorem{proposition}{Proposition}
\newtheorem{example}{Example}
\newtheorem{remark}{Remark}

% Set up the page margins
\geometry{left=0.5in,right=0.5in,top=0.5in,bottom=0.75in}

\begin{document}
\doublespacing % Use this command for double spacing
%\setstretch{3} % Use this command for triple spacing

\title{HW-04 Part TWO}
\author{Abraham J. Reines}
\date{\today}
\maketitle

\section*{4.5.21: Proposition}

Let \( n \) be an odd integer. It follows \( \lceil \frac{n}{2} \rceil = \frac{n+1}{2} \).

\section*{Proof}

\begin{enumerate}

\item \textbf{Representation of Odd Integer \( n \):}

Given \( n \) is an odd integer, we can express it as \( n = 2m + 1 \) for some integer \( m \).

\item \textbf{Evaluation of \( \lceil \frac{n}{2} \rceil \):}

Consider \( \lceil \frac{n}{2} \rceil \).

\[
\lceil \frac{2m + 1}{2} \rceil
= \lceil \frac{2m}{2} + \frac{1}{2} \rceil
= \lceil m + \frac{1}{2} \rceil
= m + 1
\]

Here, we utilize the definition and properties of the ceiling function.

\item \textbf{Expression for \( n + 1 \):}

Since \( n = 2m + 1 \), we have:

\[
n + 1 = 2m + 2 = 2(m + 1)
\]

This implies \( m + 1 = \frac{n + 1}{2} \).

\item \textbf{Final Conclusion:}

Combining these results, we find \( \lceil \frac{n}{2} \rceil = m + 1 = \frac{n + 1}{2} \).

\end{enumerate}

Thus, the proposition is proven.

\section*{4.6.22: Statement}

For any real number \( r \), if \( r^2 \) is irrational, then \( r \) is also irrational.

\section*{Proof by Contradiction}

In the contradiction, there exists a real number \( r \) such that \( r^2 \) is irrational but \( r \) is rational. We assume our premise is true: \( n \) is true and \( m \) is false.

Let \( n \) and \( m \) be integers.

Using the definition of rational numbers, \( r = \frac{n}{m} \) where \( m \neq 0 \).

Squaring both sides, we get:

\[
(r)^2 = \left( \frac{n}{m} \right)^2 \quad \Rightarrow \quad r^2 = \frac{n^2}{m^2}
\]

As products of integers, \( n^2 \) and \( m^2 \) are integers. Moreover, \( m^2 \) is non-zero.

Thus, by definition, \( \frac{n^2}{m^2} \) is rational, implying \( r^2 \) is rational. 

This contradicts the initial statement \( r^2 \) is irrational. Hence, we conclude that if \( r^2 \) is irrational, then \( r \) must also be irrational. Therefore, the statement is a contradiction.

The contradiction demonstrates our original statement is true. Thus, the proof is complete.

\newpage
\section*{Proof by Contraposition}

Let \( r \) be a real number such that \( r \) is not irrational, i.e., \( r \) is rational. Let \( n \) and \( m \) be integers. The original statement is of the form "If \( P \) then \( Q \)", where \( P \) is "\( r^2 \) is irrational" and \( Q \) is "\( r \) is irrational". The contrapositive of this statement is "If not \( Q \) then not \( P \)", which translates to "If \( r \) is not irrational (i.e., \( r \) is rational), then \( r^2 \) is not irrational (i.e., \( r^2 \) is rational)".

Using the definition of rational numbers, \( r = \frac{n}{m} \) where \( m \neq 0 \).  Squaring both sides, we get:

\[
(r)^2 = \left( \frac{n}{m} \right)^2 \quad \Rightarrow \quad r^2 = \frac{n^2}{m^2}
\]

Since \( n^2 \) and \( m^2 \) are both products of integers, they are integers. Moreover, \( m^2 \) is non-zero. Thus, by definition, \( \frac{n^2}{m^2} \) is rational, implying \( r^2 \) is rational, not irrational, which proves the contrapositive.

Therefore, the original statement "For every real number \( r \), if \( r^2 \) is irrational then \( r \) is irrational" must also be true. The proof by contraposition is thus complete.

\end{document}