\documentclass[12pt]{article}

% Packages
\usepackage{geometry}
\usepackage{setspace}
\usepackage{xcolor}
\usepackage{enumitem}
\usepackage{graphicx}
\usepackage{listings}
\usepackage{xcolor}
\usepackage{changepage}
\usepackage{endnotes}
\usepackage{amsmath, amsthm, amsfonts, amssymb}

% Custom colors for code highlighting
\definecolor{codegreen}{rgb}{0,0.6,0}
\definecolor{codegray}{rgb}{0.5,0.5,0.5}
\definecolor{codepurple}{rgb}{0.58,0,0.82}
\definecolor{backcolour}{rgb}{0.95,0.95,0.92}

% Style for Python code
\lstset{
    language=Python,
    backgroundcolor=\color{backcolour},
    commentstyle=\color{codegreen},
    keywordstyle=\color{magenta},
    numberstyle=\tiny\color{codegray},
    stringstyle=\color{codepurple},
    basicstyle=\ttfamily\scriptsize,
    breakatwhitespace=false,
    breaklines=true,
    captionpos=b,
    keepspaces=true,
    numbers=left,
    numbersep=5pt,
    showspaces=false,
    showstringspaces=false,
    showtabs=false,
    tabsize=2
}

% Define theorem, definition, and proof environments
\newtheorem{theorem}{Theorem}
\newtheorem{definition}{Definition}
\newtheorem{lemma}{Lemma}
\newtheorem{corollary}{Corollary}
\newtheorem{proposition}{Proposition}
\newtheorem{example}{Example}
\newtheorem{remark}{Remark}

% Set up the page margins
\geometry{left=0.5in,right=0.5in,top=0.5in,bottom=0.75in}

\begin{document}
\doublespacing % Use this command for double spacing

\title{HW-09}
\author{Abraham J. Reines}
\date{\today}
\maketitle

\section{9.2.15}

\begin{proposition}
A combination lock requires three selections of numbers, each from 1 through 30. We investigate two scenarios:
\begin{enumerate}
    \item How many different combinations are possible when repetition of numbers is allowed?
    \item How many different combinations are possible when no number may be used more than once?
\end{enumerate}
\end{proposition}

\begin{proof}[Solution]
\begin{enumerate}
    \item With repetition allowed, the number of combinations is calculated as the product of the number of choices for each of the three selections. Since each selection has 30 options, the total number of combinations is:
    \[ 30 \times 30 \times 30 = 27,000 \]
    
    \item Without repetition, the number of choices reduces with each selection. The first selection has 30 options, the second has 29, and the third has 28. Therefore, the total number of combinations is:
    \[ 30 \times 29 \times 28 = 24,360 \]
\end{enumerate}
\end{proof}

\section{9.2.42}

\section*{Permutation Formula Proofs}

\subsection{Proof of \( P(n+1,3) = n^3 - n \)}
\begin{theorem}
For any integer \( n \),
\[ P(n+1,3) = n^3 - n. \]
\end{theorem}

\begin{proof}
Consider the permutation formula \( nPr = \frac{n!}{(n-r)!} \), which calculates the number of ways to arrange \( r \) elements out of \( n \) distinct elements. 

Applying this to \( P(n+1, 3) \),
\[ P(n+1, 3) = \frac{(n+1)!}{(n+1-3)!} = \frac{(n+1)!}{(n-2)!}. \]

The factorial expansion yields,
\[ P(n+1, 3) = \frac{(n+1) \cdot n \cdot (n-1) \cdot (n-2)!}{(n-2)!}. \]

After simplification, we obtain
\[ P(n+1, 3) = (n+1) \cdot n \cdot (n-1) = n^3 - n. \]
\end{proof}

\subsection{Proof of \( P(n,3) = n^3 - 3n^2 + 2n \)}
\begin{theorem}
For all integers \( n \geq 3 \),
\[ P(n,3) = n^3 - 3n^2 + 2n. \]
\end{theorem}

\begin{proof}
Applying the permutation formula to \( P(n,3) \),
\[ P(n, 3) = \frac{n!}{(n-3)!} = n(n-1)(n-2). \]

Expanding this product,
\[ P(n, 3) = n(n^2 - 3n + 2) = n^3 - 3n^2 + 2n. \]
\end{proof}

\subsection{Proof of \( 3P(n,2) = 3n^2 - 3n \)}
\begin{theorem}
For any integer \( n \),
\[ 3P(n,2) = 3n^2 - 3n. \]
\end{theorem}

\begin{proof}
For \( P(n, 2) \),
\[ P(n, 2) = \frac{n!}{(n-2)!} = n(n-1). \]

Multiplying by 3,
\[ 3P(n, 2) = 3n(n-1) = 3n^2 - 3n. \]
\end{proof}

\subsection{Equality between \( P(n+1,3) - P(n,3) \) and \( 3P(n,2) \)}
\begin{theorem}
For any integer \( n \),
\[ P(n+1,3) - P(n,3) = 3P(n,2). \]
\end{theorem}

\begin{proof}
From earlier results,
\[ P(n+1,3) = n^3 - n, \]
and
\[ 3P(n,2) = 3n^2 - 3n. \]

The difference \( P(n+1,3) - P(n,3) \) is therefore
\[ (n^3 - n) - (n^3 - 3n^2 + 2n) = 3n^2 - 3n. \]

\[ \therefore P(n+1,3) - P(n,3) = 3P(n,2). \]
\end{proof}

\subsection{Conclusion}

We have systematically presented and proved several key formulas related to permutations. Through detailed mathematical proofs, we established the relationships between permutations of different sizes and their respective expansions. The proofs for \( P(n+1,3) = n^3 - n \), \( P(n,3) = n^3 - 3n^2 + 2n \), and \( 3P(n,2) = 3n^2 - 3n \) not only demonstrate the elegance of combinatorial mathematics but also highlight the interconnected nature of these permutation formulas. The final section elegantly ties together the concepts by showing the connection between \( P(n+1,3) - P(n,3) \) and \( 3P(n,2) \), further illustrating the coherence and beauty of mathematical theory in combinatorics. 

\section{9.3.24}

\subsection*{Analysis of Multiples among Integers}

In this section, we analyze the sets of integers from 1 through 1,000 that are multiples of 2 and/or 9.

Let \( A \) be the set of all integers from 1 through 1,000 that are multiples of 2, and \( B \) be the set of all such integers that are multiples of 9.

\subsection{Part a: Set Representation of Multiples of 2 or 9}
The union of sets \( A \) and \( B \), denoted \( A \cup B \), represents all integers that are multiples of either 2 or 9.

\textbf{Counting \( A \cup B \):}
\begin{itemize}
    \item \( |A| \) (Multiples of 2): There are \(\frac{1000}{2} = 500\) multiples of 2.
    \item \( |B| \) (Multiples of 9): There are \(\frac{1000}{9} = 111\) multiples of 9.
    \item \( |A \cap B| \) (Multiples of both 2 and 9): Multiples of both 2 and 9 are multiples of 18. There are \(\frac{1000}{18} = 55\) such numbers.
\end{itemize}
Using the principle of inclusion-exclusion, we find:
\[ |A \cup B| = |A| + |B| - |A \cap B| = 500 + 111 - 55 = 556. \]

\subsection{Part b: Probability of Selecting from \( A \cup B \)}
The probability of randomly selecting a multiple of 2 or 9 is the ratio of the size of \( A \cup B \) to the total number of integers:
\[ \text{Probability} = \frac{|A \cup B|}{1000} = \frac{556}{1000} = 0.556. \]

\subsection{Part c: Integers Neither Multiples of 2 Nor 9}
To find the integers that are neither multiples of 2 nor 9, we subtract \( |A \cup B| \) from the total number of integers:
\[ \text{Total integers from 1 to 1,000} = 1000. \]
\[ \text{Integers not in } A \cup B = 1000 - |A \cup B| = 1000 - 556 = 444. \]
Therefore, there are 444 integers from 1 through 1,000 that are neither multiples of 2 nor multiples of 9.

\end{document}