\documentclass[12pt]{article}

% Packages
\usepackage{amsmath}
\usepackage{amssymb}
\usepackage{amsthm}
\usepackage{geometry}
\usepackage{setspace}
\usepackage{xcolor}
\usepackage{enumitem}

% Define theorem, definition, and proof environments
\newtheorem{theorem}{Theorem}
\newtheorem{definition}{Definition}
\newtheorem{lemma}{Lemma}
\newtheorem{corollary}{Corollary}
\newtheorem{proposition}{Proposition}
\newtheorem{example}{Example}
\newtheorem{remark}{Remark}

% Set up the page margins
\geometry{left=0.5in,right=0.5in,top=0.5in,bottom=0.75in}

\begin{document}
\doublespacing % Use this command for double spacing
%\setstretch{3} % Use this command for triple spacing

\title{HW-04 Part TWO}
\author{Abraham J. Reines}
\date{\today}
\maketitle

\section{4.3.39}

\subsection{Part a: Standard Factored Form for \( a^3 \)}
Given a number \( a \) in its standard factored form:
\[
a = p_1^{e_1} \cdot p_2^{e_2} \cdot \ldots \cdot p_k^{e_k}
\]
The standard factored form for \( a^3 \) can be obtained by raising each factor in \( a \) to the third power.
\[
a^3 = (p_1^{e_1} \cdot p_2^{e_2} \cdot \ldots \cdot p_k^{e_k})^3 = p_1^{3e_1} \cdot p_2^{3e_2} \cdot \ldots \cdot p_k^{3e_k}
\]

\subsection{Part b: Least Positive Integer \( k \) for a Perfect Cube}
A number qualifies as a \textit{perfect cube} if and only if it can be expressed in the form \(a^3\), where \(a \in \mathbb{Z}\). Put simply, a number is a perfect cube if its cube root belongs to the set of integers.

\newpage

Let's scrutinize the expression \(2^4 \cdot 3^5 \cdot 7 \cdot 11^2 \cdot k\):

\begin{enumerate}[itemsep=3em, topsep=3em, partopsep=3em]
    \item The prime factor 2 manifests with an exponent of 4. To conform this factor to our perfect cube condition, the exponent should be raised to the nearest multiple of 3 greater than 4, which is 6. Consequently, \(k\) must incorporate an additional \(2^2 = 4\).
    \item The prime factor 3 is raised to the exponent 5. Following the same logic, we identify the nearest multiple of 3 greater than 5 to be 6. This necessitates the inclusion of \(3^1 = 3\) in \(k\).
    \item The prime factor 7 appears with an exponent of 1. To meet the criteria for a perfect cube, \(k\) must be supplemented with \(7^2 = 49\).
    \item The prime factor 11 is presented with an exponent of 2. The closest multiple of 3 greater than this exponent is 3, thus requiring an additional \(11^1 = 11\) in \(k\).
\end{enumerate}

To compute \(k\), we multiply these additional factors:
\[
k = 2^2 \cdot 3^1 \cdot 7^2 \cdot 11^1 = 4 \times 3 \times 49 \times 11 = 6468
\]

Therefore, multiplication of the original expression by 6468 yields a perfect cube. This elegant outcome emanates from the intrinsic properties of exponents in concert with the unique prime factorization of integers.

\newpage

\section{4.4.30}

\subsection{Part a: The Quotient-Remainder Theorem}

For any given integer \( n \) and any positive integer \( d \), there exist unique integers \( q \) and \( r \) such that:

\[
n = dq + r \quad \text{and} \quad 0 \leq r < d
\]

\subsection{Special Case: \( d = 4 \)}

Given \( d = 4 \) and any integer \( n \), the Quotient-Remainder Theorem allows us to express \( n \) in one of the following four forms:

\[
n = 4q \quad \text{or} \quad n = 4q + 1 \quad \text{or} \quad n = 4q + 2 \quad \text{or} \quad n = 4q + 3
\]

for some integer \( q \).

\subsection{Scenario 1: \( n = 4q \)}

In this scenario, the product of two consecutive integers can be expressed as:

\[
n(n + 1) = (4q)(4q + 1) = 16q^2 + 4q = 4k
\]

where \( k = 4q^2 + q \).

Given that \( q \) is an integer, it is evident that \( k = 4q^2 + q \) will also be an integer.

\subsection{Scenario 2: \( n = 4q + 1 \)}

In this case, the product of two consecutive integers can be written as:

\[
n(n + 1) = (4q + 1)(4q + 2) = 16q^2 + 12q + 2 = 4k + 2
\]

where \( k = 4q^2 + 3q \).

Again, given that \( q \) is an integer, \( k = 4q^2 + 3q \) must also be an integer.

\subsection{Scenario 3: \( n = 4q + 2 \)}

For this case, the product of two consecutive integers can be represented as:

\[
n(n + 1) = (4q + 2)(4q + 3) = 16q^2 + 20q + 6 = 4k + 2
\]

where \( k = 4q^2 + 5q + 1 \).

Since \( k \) is a sum of integers, it is necessarily an integer.

\subsection{Scenario 4: \( n = 4q + 3 \)}

In this case, the product of two consecutive integers can be written as:

\[
n(n + 1) = (4q + 3)(4q + 4) = 16q^2 + 28q + 12 = 4k
\]

where \( k = 4q^2 + 7q + 3 \).

Since \( k \) is a sum of integers, it is necessarily an integer.

\subsection{Conclusion}

In summary, the product of any two consecutive integers will have one of the forms \( 4k \) or \( 4k + 2 \), thereby demonstrating the power and utility of the Quotient-Remainder Theorem when \( d = 4 \).

\section{Part b: Using Mod Notation}
For any integer \( n \) and positive integer \( d \), \( n \) can be expressed as:

\[
n \equiv r \mod{d}
\]

where \( 0 \leq r < d \).

\newpage

\subsection{Special Case for \( d = 4 \)}

When \( d = 4 \), any integer \( n \) can be represented as one of the following forms:

\[
n \equiv 0 \mod{4} \quad \text{or} \quad n \equiv 1 \mod{4} \quad \text{or} \quad n \equiv 2 \mod{4} \quad \text{or} \quad n \equiv 3 \mod{4}
\]

\subsection{Scenario 1: \( n \equiv 0 \mod{4} \)}

The product of two consecutive integers can be expressed as:

\[
n(n+1) \equiv 0 \mod{4}
\]

\subsection{Scenario 2: \( n \equiv 1 \mod{4} \)}

The product of two consecutive integers can be expressed as:

\[
n(n+1) \equiv 2 \mod{4}
\]

\subsection{Scenario 3: \( n \equiv 2 \mod{4} \)}

The product of two consecutive integers can be expressed as:

\[
n(n+1) \equiv 2 \mod{4}
\]

\subsection{Scenario 4: \( n \equiv 3 \mod{4} \)}

The product of two consecutive integers can be expressed as:

\[
n(n+1) \equiv 0 \mod{4}
\]

\subsection{Conclusion}

In conclusion, the product of any two consecutive integers will either be congruent to \( 0 \) or \( 2 \) modulo \( 4 \), which aligns well with the Quotient-Remainder Theorem in a modular context.


\end{document}
