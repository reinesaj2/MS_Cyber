\documentclass{article}
\usepackage{amsmath}
\usepackage{amssymb} % Include this line


\begin{document}
\textbf{Name: Abraham Reines}

\section*{Question 1: Fill in the blanks}
For all objects \( T \) if \( T \) is a triangle then \( T \) has three sides.

\begin{itemize}
\item (a) All triangles have three sides.
\item (b) Every triangle has three sides.
\item (c) If an object is a triangle, then it has three sides.
\item (d) If \( T \) is a triangle, then \( T \) has three sides.
\item (e) For all triangles \( T \), \( T \) has three sides.
\end{itemize}

\section*{Question 2: Relation \( R \)}
Let \( A = \{3,5,7\} \) and \( B = \{15,16,17,18\} \), and define a relation \( R \) from \( A \) to \( B \) as follows: For all \((x,y) \in A \times B\), \((x,y) \in R \Leftrightarrow \frac{y}{x}\) is an integer.

\begin{itemize}
\item (a) Yes for \( 3 \, R \, 15 \) and \( (3,18) \in R \); No for \( 3 \, R \, 16 \) and \( (7,17) \in R \).
\item (b) \( R = \{ (3,15), (3,18), (5,15), (7,14), (7,17) \} \)
\item (c) Domain: \( A = \{3,5,7\} \), Co-domain: \( B = \{15,16,17,18\} \)
\item (d) Arrow diagram:
\begin{align*}
3 &\rightarrow 15 \\
3 &\rightarrow 18 \\
5 &\rightarrow 15 \\
7 &\rightarrow 14 \\
7 &\rightarrow 17 \\
\end{align*}
\item (e) \( R \) is not a function from \( A \) to \( B \) since 3 is related to both 15 and 18.
\end{itemize}

\section*{Question 3: Functions \( F \) and \( G \)}
Define functions \( F \) and \( G \) from \( \mathbb{R} \) to \( \mathbb{R} \) by the following formulas:
\( F(x) = (x+1)(x-3) \) and \( G(x) = (x-2)^2 - 7 \).

Answer: \( F \neq G \). They are distinct functions with different expressions.

\end{document}
